\documentclass[11pt]{article}
%Options: draft shows overfull lines, reqno-leqno puts eq numbers on right/left
%\documentclass[11pt,draft, reqno,a4paper,psamsfonts]{amsart}

\usepackage{amsmath}		% preloaded by amsart above
\usepackage{amssymb}
%\usepackage{eufrak}		% preloaded by amsart(?) above
%\usepackage[mathscr]{eucal}
\newcommand\eurfamily{\usefont{U}{eur}{m}{n}}	% Fancy script font
\DeclareTextFontCommand{\texteur}{\eurfamily}
\usepackage{amscd}		% for commutative diagrams
%\usepackage{pslatex}
%\usepackage[dvips]{epsfig}	% to include PostScript figures
\usepackage{graphicx}           % replaces epsfig
%\usepackage[dvips]{epsfig}	% to include PostScript figures
\newcommand{\comments}[1]{}     % to insert a comment
\usepackage{verbatim}
\usepackage{hyperref}
%\usepackage[active]{srcltx}     % This must be the LAST package
\usepackage{mathtools}
\DeclarePairedDelimiter\ceil{\lceil}{\rceil}
\DeclarePairedDelimiter\floor{\lfloor}{\rfloor}


% 			*** CHANGE DIMENSIONS ***
\voffset=-0.3truein		% LaTeX has too much space at page top
% \advance\hoffset by .375 in
% \advance\voffset by .125 in
\addtolength{\textheight}{0.3truein}
\addtolength{\textheight}{\topmargin}
\addtolength{\topmargin}{-\topmargin}
%\showthe\textheight
\textwidth  6.0in		% LaTeX article default 360pt=4.98''
\oddsidemargin 0pt	% \oddsidemargin  .35in   % default is 21.0 pt
\evensidemargin 0pt	% \evensidemargin .35in   % default is 59.0 pt
\mathsurround 1pt
%\parindent=20pt

% 		*** SWITCH BACK TO PLAIN.TeX USAGE ***

%The following environment is for sets of problems.
\newenvironment{problems}{\begin{list}{\arabic{numb}.}{\usecounter{numb}
\setlength{\leftmargin}{20pt}
\setlength{\labelwidth}{15pt}
\setlength{\labelsep}{5pt}
\setlength{\itemsep}{ 15.0pt plus 2.5pt minus 1.0pt}
}
}
{\end{list}}

%The next environment is for the parts a), b), ... of a problem.
 \newenvironment{parts}{\begin{list}{\alph{bean}) }{\usecounter{bean}
\setlength{\leftmargin}{20pt}
\setlength{\labelwidth}{15pt}
\setlength{\labelsep}{5pt}
\setlength{\topsep}{0pt}
\setlength{\partopsep}{0pt}
}
}
{\end{list}}

%===============Macros=========================

\renewcommand\smallskip{\vskip\smallskipamount}
\renewcommand\medskip{\vskip\medskipamount}
\renewcommand\bigskip{\vskip\bigskipamount}

% 			*** ENVIRONMENT ***
%\theoremstyle{plain}
\newtheorem{theorem}{Theorem}
\newtheorem{corollary}[theorem]{Corollary}
\newtheorem{cor}[theorem]{Corollary}
\newtheorem{lemma}[theorem]{Lemma}
\newtheorem{prop}[theorem]{Proposition}
%\theoremstyle{definition}
\newtheorem{definition}{Definition}
\newtheorem{defn}[theorem]{Definition}
%\newcommand{\Proof}{\noindent{\sc Proof:} }
\newcounter{bean}
\newcounter{bacon}
\newcounter{butter}

% 			*** FONT STUFF ***
% Font Commands (and examples):			 Example
\newcommand{\trm}{\textrm}% text roman		\trm{Rush}
\newcommand{\tbf}{\textbf}% text bold		\tbf{Rush}
\newcommand{\tit}{\textit}% text italic		\tit{Rush}
\newcommand{\tsl}{\textsl}% text slanted	  etc.
\newcommand{\ttt}{\texttt}% text typewriter
\newcommand{\tsc}{\textsc}% text small caps
\newcommand{\tsf}{\textsf}% text sans serif
\newcommand{\tup}{\textup}% text upright

\newcommand{\mrm}{\mathrm}% math roman
\newcommand{\mbf}{\mathbf}% math bold
\renewcommand{\mit}{\mathit}% math italic	% undefined by amsart
%\newcommand{\mit}{\mathit}% math italic
\newcommand{\msf}{\mathsf}% math sans serif
\newcommand{\mtt}{\mathtt}% math typewriter

\newcommand{\Cal}{\mathcal}			% Calligraphic - caps only
\newcommand{\BB}{\mathbb}			% Blackboard Bold - caps only
% the old \frak is obsolete so reuse its name
\renewcommand{\frak}{\mathfrak}			% Math Fraktur (= Gothic;
\newcommand{\BM}{\boldsymbol}			% Bold math symbol Ital
\DeclareMathAlphabet{\BMi}{OT1}{cmm}{b}{it}	% Bold math italic
\newcommand{\fancy}{\texteur}			% Fancy script

%   Examples: (assumed enclosed by $  $ )
%	\Cal{A}		Calligraphic  A
%	\BB{C}		Blackbord Bold  C
%	\frak{g}		Fraktur (= Gothic) g
%	\BM{\alpha}	bold Greek alpha
%	\BM{\Cal{A}}	bold Calligraphic  A
%	\fancy{\Large Nice}	Large size fancy:  Nice	      ($  $ not needed)
%	\BM{R}		bold math italic R
%	{\mathversion{bold} u'' = f(x)}	  prints formula bold

% 			*** MACROS ***
%MATH Macros

\newcommand{\cA}{\mathcal{A}}
\newcommand{\cB}{\mathcal{B}}
\newcommand{\cC}{\mathcal{C}}
\newcommand{\cD}{\mathcal{D}}
\newcommand{\cE}{\mathcal{E}}
\newcommand{\cF}{\mathcal{F}}
\newcommand{\cG}{\mathcal{G}}
\newcommand{\cH}{\mathcal{H}}
\newcommand{\cI}{\mathcal{I}}
\newcommand{\cJ}{\mathcal{J}}
\newcommand{\cK}{\mathcal{K}}
\newcommand{\cL}{\mathcal{L}}
\newcommand{\cM}{\mathcal{M}}
\newcommand{\cN}{\mathcal{N}}
\newcommand{\cO}{\mathcal{O}}
\newcommand{\cP}{\mathcal{P}}
\newcommand{\cQ}{\mathcal{Q}}
\newcommand{\cR}{\mathcal{R}}
\newcommand{\cS}{\mathcal{S}}
\newcommand{\cT}{\mathcal{T}}
\newcommand{\cU}{\mathcal{U}}
\newcommand{\cV}{\mathcal{V}}
\newcommand{\cW}{\mathcal{W}}
\newcommand{\cX}{\mathcal{X}}
\newcommand{\cY}{\mathcal{Y}}
\newcommand{\cZ}{\mathcal{Z}}

\newcommand{\bA}{\mathbb{A}}
\newcommand{\bB}{\mathbb{B}}
\newcommand{\bD}{\mathbb{D}}
\newcommand{\bE}{\mathbb{E}}
\newcommand{\bF}{\mathbb{F}}
\newcommand{\bG}{\mathbb{G}}
\newcommand{\bH}{\mathbb{H}}
\newcommand{\bI}{\mathbb{I}}
\newcommand{\bJ}{\mathbb{J}}
\newcommand{\bK}{\mathbb{K}}
\newcommand{\bL}{\mathbb{L}}
\newcommand{\bM}{\mathbb{M}}
\newcommand{\bN}{\mathbb{N}}
\newcommand{\bO}{\mathbb{O}}
\newcommand{\bP}{\mathbb{P}}
\newcommand{\bQ}{\mathbb{Q}}
\newcommand{\bR}{\mathbb{R}}
\newcommand{\bS}{\mathbb{S}}
\newcommand{\bT}{\mathbb{T}}
\newcommand{\bU}{\mathbb{U}}
\newcommand{\bV}{\mathbb{V}}
\newcommand{\bW}{\mathbb{W}}
\newcommand{\bX}{\mathbb{X}}
\newcommand{\bY}{\mathbb{Y}}
\newcommand{\bZ}{\mathbb{Z}}


\newcommand{\abs}[1]{\lvert #1 \rvert}		% absolute value
\newcommand{\norm}[1]{\lVert #1 \rVert}		% norm
\newcommand{\ip}[2]{\langle #1,\, #2\rangle}	% ip = inner product
\newcommand{\ipdot}[2]{ #1\cdot #2}	% ipdot = dot product
\newcommand{\im}{\text{im}}
\renewcommand{\ker}{\text{ker}}
\newcommand{\brak}[2]{[#1,\, #2]}		% brak = bracket
\newcommand{\brac}[1]{\{\,#1\,\}}		% brac = brace  {  }
\newcommand{\vect}[1]{\mathbf{#1}}		% for bold face vectors
\newcommand{\va}{\vec{a}}
\newcommand{\vb}{\vec{b}}
\newcommand{\vc}{\vec{c}}
\newcommand{\ve}{\vec{e}}
\newcommand{\vu}{\vec{u}}
\newcommand{\vv}{\vec{v}}
\newcommand{\vw}{\vec{w}}
\newcommand{\vx}{\vec{x}}
\newcommand{\vy}{\vec{y}}
\newcommand{\vz}{\vec{z}}
\newcommand{\Bb}{\mbf{b}}
\newcommand{\Be}{\mbf{e}}
\newcommand{\Bu}{\mbf{u}}
\newcommand{\Bv}{\mbf{v}}
\newcommand{\Bw}{\mbf{w}}
\newcommand{\Bx}{\mbf{x}}
\newcommand{\By}{\mbf{y}}
\newcommand{\Bz}{\mbf{z}}
%\newcommand{\Vec}[1]{\vec{#1}}
\newcommand{\bpmat}{\begin{pmatrix}}
\newcommand{\epmat}{\end{pmatrix}}
\newcommand{\PM}{\phantom{-}}
\newcommand{\setm}{\smallsetminus}
\newcommand{\cc}{,\,}
\newcommand{\bigstrut}{\rule[-5pt]{0pt}{25pt}}
\newcommand{\sstrut}{\rule[-3pt]{0pt}{15pt}}
\newcounter{numb}
\newcommand{\Numb}{\stepcounter{numb}\arabic{numb}}
\newcommand{\Numbb}{{\arabic{numb}}{\usecounter{numb}}}

\newcommand{\convd}{\overset{dis.}{\to}}
\newcommand{\convp}{\overset{pr.}{\to}}
\newcommand{\conva}{\overset{a.e.}{\to}}
\newcommand{\covlone}{\overset{$L_1$}{\to}}
\newcommand{\covltwo}{\overset{$L_2$}{\to}}
\newcommand{\covlinf}{\overset{$L_\infty$}{\to}}
\newcommand{\covlp}{\overset{$L_p$}{\to}}


\usepackage{Sweave}
\begin{document}


\Sconcordance{concordance:HW1.tex:HW1.Rnw:%
1 223 1 1 0 44 1 1 3 1 0 2 1 1 5 4 0 2 1 4 0 1 2 74 1}




\parindent=0pt
\vspace*{-20pt}
{\bf Statistics 831, Fall 2019 \hfill (Bob) Ziyang Ding}
\medskip

\begin{center}
{\large\bf Problem Set 1}

{\sc Due:} Wed. Jan. 22, 2020 
\end{center}

\begin{parts}
%=================== ????????????????????????????????????===============================

\item[1]
\textbf{$\left(\mathrm{RC} 1^{\mathrm{st}} \text { ed: } 2.5\right)$ The RANDU generator, once popular on IBM machines, is based on the recursion:
\[
X_{n+1}=65539 X_{n} \quad \bmod 2^{31}
\]
Illustrate the undesirable behavior of this generator with a computer experiment, and produce a $3 \mathrm{D}$ plot which demonstrates the problem clearly. (Hint: show $X_{t+1}=\left(6 X_{t}-9 X_{t-1}\right)$ mod $2^{31}$}

\vskip10pt
{\sc Prove:}

Note that $2^{31}=65536,$ therefore
\[
\begin{aligned}
X_{n+1} &=65539 X_{n}=\left(2^{16}-3+6\right) X_{n} \\
\left(2^{16}-3\right) X_{n} &=\left(2^{16}-3\right)\left(2^{16}+3\right) X_{n-1} \\
&=\left(2^{32}-9\right) X_{n-1} \\
&=-9 X_{n-1} \bmod 2^{31} \\
\Rightarrow X_{n+1} &=\left(6 X_{n}-9 X_{n-1}\right) \bmod 2^{31}
\end{aligned}
\]
This leads to $2^{31} | X_{n+1}-6 X_{n}+9 X_{n-1}$

Now we may proceed into making a plot:

\begin{Schunk}
\begin{Sinput}
> x = rep(0,1000)
> y = rep(0,1000)
> z = rep(0,1000)
> for(i in 1:1000){
+   x[i] = sample(1:2^(31), 1)/(2^31)
+   y[i] = (65539 * x[i]) %% 1
+   z[i] = (6*y[i] - 9*x[i]) %% (1)
+ }
> library(plotly)
> plot_ly(x=x, y=y, z=z, type="scatter3d", mode="markers", size=0.001)
\end{Sinput}
\end{Schunk}
\includegraphics{HW1-ReadOrder}


\vskip20pt
%%%%%%%%%%%%%%%%%%%%%%%%%%%%%%%%%%%%%%%%%%%%%%%%%%%%%%%%%%%%%%%%%%%%%%%%%%%

\item[2]
\textbf{In class we discussed the Box-Muller algorithm for generating pairs of iid $N(0,1)$ random variates.
\begin{enumerate}
    \item[a] Show that if $X_{1}$ and $X_{2}$ are independent standard normal random variables then their polar coordinates $R=\sqrt{X_{1}^{2}+X_{2}^{2}}$ and $\theta=$ $\tan ^{-1} X_{1} / X_{2}$ are also independent and derive their distributions.
    \item[b] Show how to generate $R$ and $\theta$ by the inverse cdf method, and a pair of standard normals via the substitution $X_{1}=R \cos \theta$ and $X_{2}=$ $R \sin \theta$
    \item[c] Implement the Box-Muller algorithm, and compare with the method described in class based on $k=12$ independent uniform $[0,1]$ r.v. $\mathrm{s}$ and the CLT. Are the two "statistically indistinguishable"? How do the respective computational costs compare?
\end{enumerate}}

\vskip10pt
{\sc Prove:}



\vskip20pt
%%%%%%%%%%%%%%%%%%%%%%%%%%%%%%%%%%%%%%%%%%%%%%%%%%%%%%%%%%%%%%%%%%%%%%%%%%%

\item[3]
\textbf{To generate the Beta distribution $\mathcal{B} e(\alpha, \beta)$ we can use the following representation:
\begin{enumerate}
    \item[a] Show that, if $Y_{1} \sim \mathcal{G} a(\alpha, 1), Y_{2} \sim \mathcal{G} a(\beta, 1),$ then
\[
X=\frac{Y_{1}}{Y_{1}+Y_{2}} \sim \mathcal{B} e(\alpha, \beta)
\]
    \item[b] Use part (a) to construct an algorithm to generate a Beta random variable.
    \item[d] Compare this algorithm with an Accept-Reject algorithm based on (i) the uniform distribution
\end{enumerate}}

\vskip10pt
{\sc Prove:}

??????????????????

\vskip20pt
%%%%%%%%%%%%%%%%%%%%%%%%%%%%%%%%%%%%%%%%%%%%%%%%%%%%%%%%%%%%%%%%%%%%%%%%%%%

\item[4]
\textbf{For the Accept-Reject algorithm [A.4], with $f$ and $g$ properly normalized,
\begin{enumerate}
    \item[a] Show that the probability of accepting a random variable is
\[
P\left(U<\frac{f(X)}{M g(X)}\right)=\frac{1}{M}
\]
    \item[b] Show that $M \geq 1$
    \item[c]  Let $N$ be the number of trials until the $k$ th random variable is accepted. Show that, for the normalized densities, $N$ has the negative binomial distribution $\mathcal{N} e g(k, p),$ where $p=1 / M .$ Deduce that the expected number of trials until $k$ random variables are obtained is $k M .$
    \item[d] Show that the bound $M$ does not have to be tight; that is, there may be $M^{\prime}<M$ such that $f(x) \leq M^{\prime} g(x) .$ Give an example where it makes sense to use $M$ instead of $M^{\prime}$
    \item[e] When the bound $M$ is too tight (i.e., when $f(x)>M g(x)$ on a non-negligible part of the support of $f),$ show that the algorithm $[A .4]$ does not produce a generation from $f .$ Give the resulting distribution.
\end{enumerate}}

\vskip10pt
{\sc Prove:}

??????????????????

\vskip20pt
%%%%%%%%%%%%%%%%%%%%%%%%%%%%%%%%%%%%%%%%%%%%%%%%%%%%%%%%%%%%%%%%%%%%%%%%%%%

\item[5]
\textbf{Given a normal distribution $N(0,1)$ restricted to $\mathbb{R}^{+},$ construct an AcceptReject algorithm based on $E x p(\lambda)$ and optimize in $\lambda .$ Repeat for intervals $[1, \infty),[2, \infty),[4, \infty) .$ For each interval, compare the efficiency to direct rejection sampling from the normal.}

\vskip10pt
{\sc Prove:}

??????????????????

\vskip20pt
%%%%%%%%%%%%%%%%%%%%%%%%%%%%%%%%%%%%%%%%%%%%%%%%%%%%%%%%%%%%%%%%%%%%%%%%%%%

\end{parts}

\end{document}
